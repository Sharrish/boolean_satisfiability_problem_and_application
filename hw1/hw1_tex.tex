\documentclass[12pt]{extreport}
\usepackage[T2A]{fontenc}
\usepackage[utf8]{inputenc}        % Кодировка входного документа;
                                    % при необходимости, вместо cp1251
                                    % можно указать cp866 (Alt-кодировка
                                    % DOS) или koi8-r.

\usepackage[english,russian]{babel} % Включение русификации, русских и
                                    % английских стилей и переносов
%%\usepackage{a4}
%%\usepackage{moreverb}
\usepackage{amsmath,amsfonts,amsthm,amssymb,amsbsy,amstext,amscd,amsxtra,multicol}
\usepackage{indentfirst}
\usepackage{verbatim}
\usepackage{tikz} %Рисование автоматов
\usetikzlibrary{automata,positioning}
\usepackage{multicol} %Несколько колонок
\usepackage{graphicx}
\usepackage[colorlinks,urlcolor=blue]{hyperref}
\usepackage[stable]{footmisc}

\usepackage[linesnumbered]{algorithm2e}   

%% \voffset-5mm
%% \def\baselinestretch{1.44}
\renewcommand{\theequation}{\arabic{equation}}
\def\hm#1{#1\nobreak\discretionary{}{\hbox{$#1$}}{}}
\newtheorem{Lemma}{Лемма}
\theoremstyle{definiton}
\newtheorem{Remark}{Замечание}
%%\newtheorem{Def}{Определение}
\newtheorem{Claim}{Утверждение}
\newtheorem{Cor}{Следствие}
\newtheorem{Theorem}{Теорема}
\theoremstyle{definition}
\newtheorem{Example}{Пример}
\newtheorem*{known}{Теорема}
\def\proofname{Доказательство}
\theoremstyle{definition}
\newtheorem{Def}{Определение}

%% \newenvironment{Example} % имя окружения
%% {\par\noindent{\bf Пример.}} % команды для \begin
%% {\hfill$\scriptstyle\qed$} % команды для \end






%\date{22 июня 2011 г.}
\let\leq\leqslant
\let\geq\geqslant
\def\MT{\mathrm{MT}}
%Обозначения ``ажуром''
\def\BB{\mathbb B}
\def\CC{\mathbb C}
\def\RR{\mathbb R}
\def\SS{\mathbb S}
\def\ZZ{\mathbb Z}
\def\NN{\mathbb N}
\def\FF{\mathbb F}
%греческие буквы
\let\epsilon\varepsilon
\let\es\varnothing
\let\eps\varepsilon
\let\al\alpha
\let\sg\sigma
\let\ga\gamma
\let\ph\varphi
\let\om\omega
\let\ld\lambda
\let\Ld\Lambda
\let\vk\varkappa
\let\Om\Omega
\def\abstractname{}

\def\R{{\cal R}}
\def\A{{\cal A}}
\def\B{{\cal B}}
\def\C{{\cal C}}
\def\D{{\cal D}}

%классы сложности
\def\REG{{\mathsf{REG}}}
\def\CFL{{\mathsf{CFL}}}


%%%%%%%%%%%%%%%%%%%%%%%%%%%%%%% Problems macros  %%%%%%%%%%%%%%%%%%%%%%%%%%%%%%%


%%%%%%%%%%%%%%%%%%%%%%%% Enumerations %%%%%%%%%%%%%%%%%%%%%%%%

\newcommand{\Rnum}[1]{\expandafter{\romannumeral #1\relax}}
\newcommand{\RNum}[1]{\uppercase\expandafter{\romannumeral #1\relax}}

%%%%%%%%%%%%%%%%%%%%% EOF Enumerations %%%%%%%%%%%%%%%%%%%%%

\usepackage{xparse}
\usepackage{ifthen}
\usepackage{bm} %%% bf in math mode
\usepackage{color}
%\usepackage[usenames,dvipsnames]{xcolor}

\definecolor{Gray555}{HTML}{555555}
\definecolor{Gray444}{HTML}{444444}
\definecolor{Gray333}{HTML}{333333}


\newcounter{problem}
\newcounter{uproblem}
\newcounter{subproblem}
\newcounter{prvar}

\def\beforPRskip{
	\bigskip
	%\vspace*{2ex}
}

\def\PRSUBskip{
	\medskip
}


\def\pr{\beforPRskip\noindent\stepcounter{problem}{\bf \theproblem .\;}\setcounter{subproblem}{0}}
\def\pru{\beforPRskip\noindent\stepcounter{problem}{\bf $\mathbf{\theproblem}^\circ$\!\!.\;}\setcounter{subproblem}{0}}
\def\prstar{\beforPRskip\noindent\stepcounter{problem}{\bf $\mathbf{\theproblem}^*$\negthickspace.}\setcounter{subproblem}{0}\;}
\def\prpfrom[#1]{\beforPRskip\noindent\stepcounter{problem}{\bf Задача \theproblem~(№#1 из задания).  }\setcounter{subproblem}{0} }
\def\prp{\beforPRskip\noindent\stepcounter{problem}{\bf Задача \theproblem .  }\setcounter{subproblem}{0} }

\def\prpvar{\beforPRskip\noindent\stepcounter{problem}\setcounter{prvar}{1}{\bf Задача \theproblem \;$\langle${\rm\Rnum{\theprvar}}$\rangle$.}\setcounter{subproblem}{0}\;}
\def\prpv{\beforPRskip\noindent\stepcounter{prvar}{\bf Задача \theproblem \,$\bm\langle$\bracketspace{{\rm\Rnum{\theprvar}}}$\bm\rangle$.  }\setcounter{subproblem}{0} }
\def\prv{\beforPRskip\noindent\stepcounter{prvar}{\bf \theproblem\,$\bm\langle$\bracketspace{{\rm\Rnum{\theprvar}}}$\bm\rangle$}.\setcounter{subproblem}{0} }

\def\prpstar{\beforPRskip\noindent\stepcounter{problem}{\bf Задача $\bf\theproblem^*$\negthickspace.  }\setcounter{subproblem}{0} }
\def\prdag{\beforPRskip\noindent\stepcounter{problem}{\bf Задача $\theproblem^{^\dagger}$\negthickspace\,.  }\setcounter{subproblem}{0} }
\def\upr{\beforPRskip\noindent\stepcounter{uproblem}{\bf Упражнение \theuproblem.}\setcounter{subproblem}{0}\;}
%\def\prp{\vspace{5pt}\stepcounter{problem}{\bf Задача \theproblem .  } }
%\def\prs{\vspace{5pt}\stepcounter{problem}{\bf \theproblem .*   }
\def\prsub{\PRSUBskip\noindent\stepcounter{subproblem}{\sf \thesubproblem.}\;}
\def\prsubr{\PRSUBskip\noindent\stepcounter{subproblem}{\bf \asbuk{subproblem})}\;}
\def\prsubstar{\PRSUBskip\noindent\stepcounter{subproblem}{\rm $\thesubproblem^*$\negthickspace.}\;}
\def\prsubrstar{\PRSUBskip\noindent\stepcounter{subproblem}{$\text{\bf \asbuk{subproblem}}^*\mathbf{)}$}\;}

\newcommand{\bracketspace}[1]{\phantom{(}\!\!{#1}\!\!\phantom{)}}

\DeclareDocumentCommand{\Prpvar}{ O{null} O{} }{
	\beforPRskip\noindent\stepcounter{problem}\setcounter{prvar}{1}{\bf Задача \theproblem
% 	\ifthenelse{\equal{#1}{null}}{  }{ {\sf $\bm\langle$\bracketspace{#1}$\bm\rangle$}}
%	~\!\!(\bracketspace{{\rm\Rnum{\theprvar}}}).  }\setcounter{subproblem}{0}
%	\;(\bracketspace{{\rm\Rnum{\theprvar}}})}\setcounter{subproblem}{0}
%
	\,{\sf $\bm\langle$\bracketspace{{\rm\Rnum{\theprvar}}}$\bm\rangle$}
	~\!\!\! \ifthenelse{\equal{#1}{null}}{\!}{{\sf(\bracketspace{#1})}}}.

}
%\DeclareDocumentCommand{\Prpvar}{ O{level} O{meta} m }{\prpvar}


\DeclareDocumentCommand{\Prp}{ O{null} O{null} }{\setcounter{subproblem}{0}
	\beforPRskip\noindent\stepcounter{problem}\setcounter{prvar}{0}{\bf Задача \theproblem
	~\!\!\! \ifthenelse{\equal{#1}{null}}{\!}{{\sf(\bracketspace{#1})}}
	 \ifthenelse{\equal{#2}{null}}{\!\!}{{\sf [\color{Gray444}\,\bracketspace{{\fontfamily{afd}\selectfont#2}}\,]}}}.}

\DeclareDocumentCommand{\Pr}{ O{null} O{null} }{\setcounter{subproblem}{0}
	\beforPRskip\noindent\stepcounter{problem}\setcounter{prvar}{0}{\bf\theproblem
	~\!\! \ifthenelse{\equal{#1}{null}}{\!\!}{{\sf(\bracketspace{#1})}}
	 \ifthenelse{\equal{#2}{null}}{\!\!}{{\sf [\color{Gray444}\,\bracketspace{{\fontfamily{afd}\selectfont#2}}\,]}}}.}
	
	\DeclareDocumentCommand{\Prstar}{ O{null} O{null} }{\setcounter{subproblem}{0}
			\medskip\noindent\stepcounter{problem}\setcounter{prvar}{0}{\bf$\mathbf{\theproblem^*}$
			~\!\!\! \ifthenelse{\equal{#1}{null}}{\!}{{\sf(\bracketspace{#1})}}
			 \ifthenelse{\equal{#2}{null}}{\!\!}{{\sf [\color{Gray444}\,\bracketspace{{\fontfamily{afd}\selectfont#2}}\,]}}}.}
	

%\DeclareDocumentCommand{\Prp}{ O{level} O{meta} }

\DeclareDocumentCommand{\Prps}{ O{null} O{null} }{\setcounter{subproblem}{0}
	\beforPRskip\noindent\stepcounter{problem}\setcounter{prvar}{0}{\bf Задача $\bm\theproblem^* $
	~\!\!\! \ifthenelse{\equal{#1}{null}}{\!}{{\sf(\bracketspace{#1})}}
	 \ifthenelse{\equal{#2}{null}}{\!\!}{{\sf [\color{Gray444}\,\bracketspace{{\fontfamily{afd}\selectfont#2}}\,]}}}.
}

\DeclareDocumentCommand{\Prpd}{ O{null} O{null} }{\setcounter{subproblem}{0}
	\beforPRskip\noindent\stepcounter{problem}\setcounter{prvar}{0}{\bf Задача $\bm\theproblem^\dagger$
	~\!\!\! \ifthenelse{\equal{#1}{null}}{\!}{{\sf(\bracketspace{#1})}}
	 \ifthenelse{\equal{#2}{null}}{\!\!}{{\sf [\color{Gray444}\,\bracketspace{{\fontfamily{afd}\selectfont#2}}\,]}}}.
}


\def\prend{
	\bigskip
%	\bigskip
}




%%%%%%%%%%%%%%%%%%%%%%%%%%%%%%% EOF Problems macros  %%%%%%%%%%%%%%%%%%%%%%%%%%%%%%%



%\usepackage{erewhon}
%\usepackage{heuristica}
%\usepackage{gentium}

\usepackage[portrait, top=3cm, bottom=1.5cm, left=3cm, right=2cm]{geometry}

\usepackage{fancyhdr}
\pagestyle{fancy}
% \renewcommand{\headrulewidth}{0pt}
\lhead{\fontfamily{fca}\selectfont {\color{myblue}\bf{Шарипов Саит}} }
\rhead{\fontfamily{fca}\selectfont {Задачи разрешимости логических формул и приложения\ \ \ \  ДЗ№1} }
% \rhead{\fontfamily{fca}\selectfont ДЗ№1}
%\lhead{ \bf  {ТРЯП. } Семинар 1 }
%\chead{\fontfamily{fca}\selectfont {Вариант 1}}
%\rhead{\small 01.09.2016}
\cfoot{}

\usepackage{titlesec}
\titleformat{\section}[block]{\Large\bfseries\filcenter {\setcounter{problem}{0}}  }{}{1em}{}


%%%%%%%%%%%%%%%%%%%%%%%%%%%%%%%%%%%%%%%%%%%%%%%%%%%% Обозначения и операции %%%%%%%%%%%%%%%%%%%%%%%%%%%%%%%%%%%%%%%%%%%%%%%%%%%% 
                                                                    
\newcommand{\divisible}{\mathop{\raisebox{-2pt}{\vdots}}}           
\let\Om\Omega


%%%%%%%%%%%%%%%%%%%%%%%%%%%%%%%%%%%%%%%% Shen Macroses %%%%%%%%%%%%%%%%%%%%%%%%%%%%%%%%%%%%%%%%
\newcommand{\w}[1]{{\hbox{\texttt{#1}}}}
\usepackage{wrapfig}
\def\bin{\mathrm{bin}}

% МОЙ КОД НАЧАЛО
\definecolor{myblue}{RGB}{0, 0, 102}
\definecolor{myyellow}{RGB}{252, 245, 174}

\newcommand{\solution}[2][\color{myblue}Решение]{
\medskip
	\noindent{\bfseries #1 }{{\color{myblue}\bfseries #2:}}
%\medskip	
}

\newenvironment{blockquote}{%
  \par%
  \medskip
  \leftskip=1em%
  \noindent}{%
  \par\medskip}
  
\usepackage{listings}
\usepackage{xcolor}
\lstset { %
    language=C++,
    belowcaptionskip=1\baselineskip,
    breaklines=true,
    frame=L,
    xleftmargin=\parindent,
    language=C,
    showstringspaces=false,
    basicstyle=\footnotesize\ttfamily,
    keywordstyle=\bfseries\color{purple!40!black},
    commentstyle=\itshape\color{green!40!black},
    identifierstyle=\color{blue},
    stringstyle=\color{orange},
    backgroundcolor=\color{black!5}, % set backgroundcolor
    basewidth=0.5em,
    numbers=left,
}
% МОЙ КОД КОНЕЦ



\begin{document}	
\SetKwFunction{BuildMaxHeap}{Build\_Max\_Heap} 
\SetKwFunction{TreeSuccessor}{Tree\_Successor} 
\SetKwFunction{ExtMax}{Extract\_Max} 
\SetKwFunction{MaxHeapify}{Max\_Heapify}

\SetKwInOut{Input}{Вход}\SetKwInOut{Output}{Выход}
\SetKwProg{Fn}{Function}{ :}{end}
\SetKwFunction{F}{F}
\SetKwFunction{KWsize}{size}
\SetKwFunction{HeapSize}{heap\_size}
\SetKwFunction{KwPrint}{print} 
\SetKwFunction{swap}{swap}
			
\Pr[10 баллов] Даны булевы переменные $x_1, \dots, x_5$. Постройте две разные булевы формулы в форме КНФ, которые выражают следующую формулу: $x_1 + \dots + x_5 \leq 2$. Первая должна содержать только переменные $x_1, \dots, x_5$, вторая -- дополнительные переменные.
			
	\solution{1}
	\begin{blockquote}
	{\color{myblue}
	\noindent \textbf{1.} В первой формуле нужно реализовать $AtMostTwo$.\\
	Общая формула для $AtMostK$ выглядит так:\\
	$${\bigwedge\limits_{\substack{1 \leq i_1, \dots, i_{k+1} \leq n \\
	i_h \neq i_m}} (\overline{x_{i_1}} \vee \overline{x_{i_2}} \vee \ldots \vee \overline{x_{i_{k+1}}})}$$
	Тогда формула $\phi_1 = AtMostTwo$ для $x_1, \dots, x_5$ будет выглядеть так:
	$$\phi_1 = AtMostTwo = {\bigwedge\limits_{\substack{1 \leq i, j, k \leq 5 \\
	i \neq j \neq k}} (\overline{x_{i}} \vee \overline{x_{j}} \vee \overline{x_{k}})}$$
	\\
	\noindent \textbf{2.} Введем дополнительные переменные и получим формулу $\phi_2$, т.ч. она будет равновыполнима формуле $\phi_1$.
	$$\phi_2 = AtMostTwo(x_1, x_2, x_3, z_1, z_2) \wedge AtMostTwo(\overline{z_1}, \overline{z_2}, x_4, x_5)$$
	Тут есть $AtMostTwo$ от четырех переменных -- она строится аналогично как и от пяти переменных.\\
	Заметим, что если формула $\phi_2$ истина, то и формула $\phi_1$ истина (равносильно тому, что $\phi_1 = 0 \Rightarrow \phi_2 = 0$):
	\begin{itemize}
	    \item Если $z_1 = 0, z_2 = 0$, то $x_1 + x_2 + x_3 \leq 2$ и $x_4 + x_5 = 0 \Rightarrow x_1 + x_2 + x_3 + x_4 + x_5 \leq 2$;
	    \item Если $z_1 = 0, z_2 = 1$, то $x_1 + x_2 + x_3 \leq 1$ и $x_4 + x_5 \leq 1 \Rightarrow x_1 + x_2 + x_3 + x_4 + x_5 \leq 2$;
	    \item Если $z_1 = 1, z_2 = 0$, то $x_1 + x_2 + x_3 \leq 1$ и $x_4 + x_5 \leq 1 \Rightarrow x_1 + x_2 + x_3 + x_4 + x_5 \leq 2$;
	    \item Если $z_1 = 1, z_2 = 1$, то $x_1 + x_2 + x_3 = 0$ и $x_4 + x_5 \leq 2 \Rightarrow x_1 + x_2 + x_3 + x_4 + x_5 \leq 2$;
	\end{itemize}
	При этом также, если формула $\phi_1$ истина, то и формула $\phi_2$ истина при некоторых $z_1$, $z_2$. Обозначим $s_1 = x_1 + x_2 + x_3$, $s_2 = x_4 + x_5$. В условиях истинности формулы $\phi_1$ -- $s_1 + s_2 \leq 2$ и возможны следующие варианты:
	\begin{itemize}
	    \item Если $s_1 = 0$, то при любом $s_2 \in \{0, 1, 2\}$, при $z_1 = 1, z_2 = 1 \Rightarrow \phi_2$ истина;
	    \item Если $s_2 = 0$, то при любом $s_1 \in \{0, 1, 2\}$, при $z_1 = 0, z_2 = 0 \Rightarrow \phi_2$ истина;
	    \item Если $s_1 = 1$ и $s_2 = 1$, то при $z_1 = 0, z_2 = 1 \Rightarrow \phi_2$ истина;
	\end{itemize}
	Таким образом мы показали, что формулы действительно равновыполнимы.
	}
	\end{blockquote}

\Pr[10 баллов] Обозначим первую формулу как $AtMostTwoA$, вторую -- $AtMostTwoB$. Представьте в виде КНФ формулы\\$y_1 \leftrightarrow AtMostTwoA(x_1, \dots, x_5)$, $y_2 \leftrightarrow AtMostTwoB(x_1, \dots, x_5)$.

    \solution{2}
    \begin{blockquote}
    {\color{myblue}
    \noindent \textbf{1. КНФ для формулы $y_1 \leftrightarrow AtMostTwoA(x_1, \dots, x_5)$.}\\
    $$x \leftrightarrow y \Leftrightarrow (x \vee \overline{y}) \wedge (\overline{x} \vee y)$$
    Следовательно:\\
    $$y_1 \leftrightarrow AtMostTwoA(x_1, \dots, x_5)$$
    $$(y_1 \vee \overline{AtMostTwoA(x_1, \dots, x_5)}) \wedge (\overline{y_1} \vee AtMostTwoA(x_1, \dots, x_5))$$
    $$(y_1 \vee {\bigwedge\limits_{\substack{1 \leq i, j, k \leq 5 \\
	i \neq j \neq k}} (x_{i} \vee x_{j} \vee x_{k}})) \wedge (\overline{y_1} \vee {\bigwedge\limits_{\substack{1 \leq i, j, k \leq 5 \\
	i \neq j \neq k}} (\overline{x_{i}} \vee \overline{x_{j}} \vee \overline{x_{k}})})$$
	$$CNF = \big({\bigwedge\limits_{\substack{1 \leq i, j, k \leq 5 \\
	i \neq j \neq k}} (x_{i} \vee x_{j} \vee x_{k} \vee y_1)}\big)  \wedge \big({\bigwedge\limits_{\substack{1 \leq i, j, k \leq 5 \\
	i \neq j \neq k}} (\overline{x_{i}} \vee \overline{x_{j}} \vee \overline{x_{k} } \vee \overline{y_1})}\big)$$
	В файле {\fcolorbox{red}{myyellow}{task2\_AtMostTwoA.py}} находится программа, которая построчно печатает все дизьюнкты КНФ данной формулы.\\
	\\
    \textbf{2. КНФ для формулы $y_2 \leftrightarrow AtMostTwoB(x_1, \dots, x_5)$.}\\
    Для краткости обозначим $f$ как $AtMostTwo$:\\
    $$y_2 \leftrightarrow AtMostTwoB(x_1, \dots, x_5)$$
    $$y_2 \leftrightarrow \big(f(x_1, x_2, x_3, z_1, z_2) \wedge f(\overline{z_1}, \overline{z_2}, x_4, x_5)\big)$$
    $$(y_2 \vee \overline{f(x_1, x_2, x_3, z_1, z_2)} \vee \overline{f(\overline{z_1}, \overline{z_2}, x_4, x_5)}) \wedge (\overline{y_2} \vee \big(f(x_1, x_2, x_3, z_1, z_2) \wedge f(\overline{z_1}, \overline{z_2}, x_4, x_5)\big))$$
    Для удобства обозначим через $h_1 = x_1, h_2 = x_2, h_3 = x_3, h_4 = z_1, h_5 = z_2$ и $g_1 = \overline{z_1}, g_2 = \overline{z_2}, g_3 = x_3, g_4 = x_4$.\\
    $$y_2 \vee \overline{f(x_1, x_2, x_3, z_1, z_2)} \vee \overline{f(\overline{z_1}, \overline{z_2}, x_4, x_5)} = y_2 \vee {\bigwedge\limits_{\substack{1 \leq i, j, k \leq 5 \\
	i \neq j \neq k}} (h_i \vee h_j \vee h_k)} \vee {\bigwedge\limits_{\substack{1 \leq i, j, k \leq 4 \\
	i \neq j \neq k}} (g_i \vee g_j \vee g_k)} = $$
	$$= {\bigwedge\limits_{\substack{1 \leq i, j, k \leq 5 \\
	i \neq j \neq k}}} \ {\bigwedge\limits_{\substack{1 \leq n, l, m \leq 5 \\ n \neq l \neq m}}} (y_2 \vee h_i \vee h_j \vee h_k \vee g_n \vee g_l \vee g_m)$$
	$$\overline{y_2} \vee \big(f(x_1, x_2, x_3, z_1, z_2) \wedge f(\overline{z_1}, \overline{z_2}, x_4, x_5)\big) = \overline{y_2} \vee \big({\bigwedge\limits_{\substack{1 \leq i, j, k \leq 5 \\
	i \neq j \neq k}} (\overline{h_{i}} \vee \overline{h_{j}} \vee \overline{h_{k}})} \wedge {\bigwedge\limits_{\substack{1 \leq i, j, k \leq 4 \\
	i \neq j \neq k}} (\overline{g_{i}} \vee \overline{g_{j}} \vee \overline{g_{k}})}\big) =$$
	$$= {\bigwedge\limits_{\substack{1 \leq i, j, k \leq 5 \\
	i \neq j \neq k}} (\overline{h_{i}} \vee \overline{h_{j}} \vee \overline{h_{k}} \vee \overline{y_2})} \wedge {\bigwedge\limits_{\substack{1 \leq i, j, k \leq 4 \\
	i \neq j \neq k}} (\overline{g_{i}} \vee \overline{g_{j}} \vee \overline{g_{k}}  \vee \overline{y_2})}$$
	Таким образом:\\
	$$CNF = {\bigwedge\limits_{\substack{1 \leq i, j, k \leq 5 \\
	i \neq j \neq k}}} \ {\bigwedge\limits_{\substack{1 \leq n, l, m \leq 5 \\ n \neq l \neq m}}} (y_2 \vee h_i \vee h_j \vee h_k \vee g_n \vee g_l \vee g_m) \wedge$$
	$$\wedge \big({\bigwedge\limits_{\substack{1 \leq i, j, k \leq 5 \\
	i \neq j \neq k}} (\overline{h_{i}} \vee \overline{h_{j}} \vee \overline{h_{k}} \vee \overline{y_2})} \wedge {\bigwedge\limits_{\substack{1 \leq i, j, k \leq 4 \\
	i \neq j \neq k}} (\overline{g_{i}} \vee \overline{g_{j}} \vee \overline{g_{k}}  \vee \overline{y_2})}\big)$$
	Так это делается теоретически, однако довольно просто можно найти совершенную КНФ по самой формуле. Минус такого подхода в том, что число дизъюнктов может быть больше, чем в теоретическом подходе. В качестве разнообразия в программе будем использовать СКНФ данной формулы.\\
	В файле {\fcolorbox{red}{myyellow}{task2\_AtMostTwoB.py}} находится программа, которая построчно печатает все дизьюнкты СКНФ данной формулы.\\
    }
    \end{blockquote}

\Pr[5 баллов] Напишите формулу, выражающую следующее утверждение: \textit{"существует оценка $x_1, \dots, x_5$, такая, что не выполняется $y_1$ и выполняется $y_2$"}, используя $y_1 \leftrightarrow AtMostTwoA(x_1, \dots, x_5)$ и $y_2 \leftrightarrow AtMostTwoB(x_1, \dots, x_5)$.

    \solution{3}
    \begin{blockquote}
    {\color{myblue}
    \noindent Тут подход похож (а может быть это он и есть) на \textit{преобразование Цейтина}.\\
    \\
    $T_1 \leftrightarrow \overline{y_1} \wedge y_2$\\
    $y_1 \leftrightarrow AtMostTwoA(x_1, \dots, x_5)$\\
    $y_2 \leftrightarrow AtMostTwoB(x_1, \dots, x_5)$\\
    \\
    Пусть $F_1$ -- КНФ формулы $y_1 \leftrightarrow AtMostTwoA(x_1, \dots, x_5)$,\\
    $F_2$ -- КНФ формулы $y_2 \leftrightarrow AtMostTwoB(x_1, \dots, x_5)$.\\
    \\
    Тогда формула, выражающая утверждение есть:
    $T_1 \wedge F_1 \wedge F_2 = \overline{y_1} \wedge y_2 \wedge F_1 \wedge F_2$.\\
    Осталось заметить, что КНФ $F_1$ и $F_2$ были построены в предыдущем номере.\\
    \\
    \textbf{Ответ:} $\overline{y_1} \wedge y_2 \wedge F_1 \wedge F_2$.
    }
    \end{blockquote}

\Pr[5 баллов] Проверьте получившуюся формулу на SAT-решателе и предоставьте результат вычисления.

    \solution{4}
    \begin{blockquote}
    {\color{myblue}
    \noindent Прежде, чем программировать, подумаем, может ли быть выполнима формула, которая выражает утверждение \textit{"существует оценка $x_1, \dots, x_5$, такая, что не выполняется $y_1$ и выполняется $y_2$"}? Хммм, в $1$-ом номере мы проверяли, что, если истина формула $\phi_2$, то будет верна и формула $\phi_1$. Поэтому небольшой спойлер: SAT-решатель выдаст вердикт, что формула UNSAT (\textit{unsatisfiable}).\\
    \\
    Что же, все действительно так. При запуске программы {\fcolorbox{red}{myyellow}{task4.py}} выводится слово \textbf{\textit{"UNSAT"}}.
    }
    \end{blockquote}

\Pr[10 баллов] Рассмотрим квадратную таблицу $10 \times 10$ и все возможные прямоугольники внутри сетки, длина и ширина которых не менее $2$. Напишите следующую формулу: \textit{"существует ли раскраска сетки с использованием трех цветов, чтобы ни один такой прямоугольник не имел одинакового цвета в своих четырех углах"}.

    \solution{5}
    \begin{blockquote}
    {\color{myblue}
    \noindent Введем следующие обозначения:\\
    $x_{ij1}$ -- истина тогда и только тогда, когда клетка $i \times j$ раскрашена в цвет $1$.\\
    $x_{ij2}$ -- истина тогда и только тогда, когда клетка $i \times j$ раскрашена в цвет $2$.\\
    $x_{ij3}$ -- истина тогда и только тогда, когда клетка $i \times j$ раскрашена в цвет $3$.\\
    Таким образом у нас всего $10 \cdot 10 \cdot 3 = 300$ переменных.\\
    Нужно проследить за тем, чтобы каждая клетка была раскрашена ровно в один из трех цветов. Это делает следующая формула:
    $$\bigwedge\limits_{\substack{1 \leq i, j \leq 10}} \Big((x_{ij1} \vee x_{ij2} \vee x_{ij3}) \wedge AtMostOne(x_{ij1}, x_{ij2}, x_{ij3})\Big)= \bigwedge\limits_{\substack{1 \leq i, j \leq 10}} \Big((x_{ij1} \vee x_{ij2} \vee x_{ij3}) \wedge \bigwedge\limits_{\substack{1 \leq k, l \leq 3 \\ k \neq l}} (\overline{x_{ijk}} \vee \overline{x_{ijl}}) \Big)$$
    Теперь таблица раскрашена правильно. Далее надо обеспечить, чтобы углы ни одного из прямоугольников с длиной и шириной $\geq 2$ не имел бы $4$-х углов одного и того же цвета.
    Обозначим углы какого-либо прямоугольника следующим образом:
    \begin{center}
    \begin{tikzpicture}[scale=0.2]
    \tikzstyle{every node}+=[inner sep=0pt]
    \draw [black] (32.1,-12.2) circle (3);
    \draw (32.1,-12.2) node {$a_1,b_2$};
    \draw [black] (18.5,-25.1) circle (3);
    \draw (18.5,-25.1) node {$a_2,b_1$};
    \draw [black] (32.1,-25.1) circle (3);
    \draw (32.1,-25.1) node {$a_2,b_2$};
    \draw [black] (18.5,-12.2) circle (3);
    \draw (18.5,-12.2) node {$a_1,b_1$};
    \draw [black] (21.5,-12.2) -- (29.1,-12.2);
    \fill [black] (29.1,-12.2) -- (28.3,-11.7) -- (28.3,-12.7);
    \draw [black] (32.1,-15.2) -- (32.1,-22.1);
    \fill [black] (32.1,-22.1) -- (32.6,-21.3) -- (31.6,-21.3);
    \draw [black] (29.1,-25.1) -- (21.5,-25.1);
    \fill [black] (21.5,-25.1) -- (22.3,-25.6) -- (22.3,-24.6);
    \draw [black] (18.5,-21.9) -- (18.5,-15.2);
    \fill [black] (18.5,-15.2) -- (18,-16) -- (19,-16);
    \end{tikzpicture}
    \end{center}
    Следующая формула обеспечивает то, что его углы не будут одного цвета:
    $$\overline{\bigvee\limits_{\substack{1 \leq k \leq 3}}x_{a_1b_1k} \wedge x_{a_1b_2k} \wedge x_{a_2b_1k} \wedge x_{a_2b_2k}} = \bigwedge\limits_{\substack{1 \leq k \leq 3}}\overline{x_{a_1b_1k}} \vee \overline{x_{a_1b_2k}} \vee \overline{x_{a_2b_1k}} \vee \overline{x_{a_2b_2k}}$$
    Распространим это требование ко всем прямоугольникам в таблице:
    $$\bigwedge\limits_{\substack{1 \leq a_1, b_1, a_2, b_2 \leq 10\\a_2 - a_1 \geq 2 \\b_2 - b_1 \geq 2}}\Big(\bigwedge\limits_{\substack{1 \leq k \leq 3}}\overline{x_{a_1b_1k}} \vee \overline{x_{a_1b_2k}} \vee \overline{x_{a_2b_1k}} \vee \overline{x_{a_2b_2k}}\Big)$$
    Осталось собрать вместе все наши требования:
    $$\Bigg(\bigwedge\limits_{\substack{1 \leq i, j \leq 10}} \Big((x_{ij1} \vee x_{ij2} \vee x_{ij3}) \wedge \bigwedge\limits_{\substack{1 \leq k, l \leq 3 \\ k \neq l}} (\overline{x_{ijk}} \vee \overline{x_{ijl}}) \Big)\Bigg) \wedge \Bigg(\bigwedge\limits_{\substack{1 \leq a_1, b_1, a_2, b_2 \leq 10\\a_2 - a_1 \geq 2 \\b_2 - b_1 \geq 2}}\Big(\bigwedge\limits_{\substack{1 \leq k \leq 3}}\overline{x_{a_1b_1k}} \vee \overline{x_{a_1b_2k}} \vee \overline{x_{a_2b_1k}} \vee \overline{x_{a_2b_2k}}\Big)\Bigg)$$

    }
    \end{blockquote}

\Pr[10 баллов] Решите формулу на SAT-решателе и по результату решателя (если формула разрешима) постройте нужную таблицу.

    \solution{6}
    \begin{blockquote}
    {\color{myblue}
    \noindent Формула разрешима! Запуск программы {\fcolorbox{red}{myyellow}{task6.py}} выводит требуемую таблицу.
    }
    \end{blockquote}
			
			
\end{document}
  